 \documentclass[12pt]{article}
\usepackage{graphicx}
\usepackage{booktabs}
 \usepackage{makecell}
 \usepackage{float}
 \newcommand{\diff}{\,\mathrm{d}}
\usepackage[margin=1in]{geometry}
\usepackage{fancyhdr}
\pagestyle{fancy}
\usepackage{extarrows}
\usepackage{breqn}

\newcommand{\N}{\mathbb{N}}
\newcommand{\Z}{\mathbb{Z}}
\newcommand{\trans}{^{\mathrm T}}
\usepackage{amssymb}
\usepackage[table]{xcolor}
\usepackage{bm}
\usepackage{array}
\usepackage{mathtools}
\usepackage[english]{babel}
\usepackage{natbib}
\usepackage{url}
\usepackage[utf8x]{inputenc}
\usepackage{amsmath}
\usepackage{graphicx}
\graphicspath{{images/}}
\usepackage{parskip}
\usepackage{fancyhdr}
\usepackage{vmargin}
\usepackage[font={bf, footnotesize}, textfont=md]{caption}
\usepackage{amsmath,amsthm,amssymb}


\newenvironment{theorem}[2][Theorem]{\begin{trivlist}
\item[\hskip \labelsep {\bfseries #1}\hskip \labelsep {\bfseries #2.}]}{\end{trivlist}}
\newenvironment{lemma}[2][Lemma]{\begin{trivlist}
\item[\hskip \labelsep {\bfseries #1}\hskip \labelsep {\bfseries #2.}]}{\end{trivlist}}
\newenvironment{exercise}[2][Exercise]{\begin{trivlist}
\item[\hskip \labelsep {\bfseries #1}\hskip \labelsep {\bfseries #2.}]}{\end{trivlist}}
\newenvironment{reflection}[2][Reflection]{\begin{trivlist}
\item[\hskip \labelsep {\bfseries #1}\hskip \labelsep {\bfseries #2.}]}{\end{trivlist}}
\newenvironment{proposition}[2][Proposition]{\begin{trivlist}
\item[\hskip \labelsep {\bfseries #1}\hskip \labelsep {\bfseries #2.}]}{\end{trivlist}}
\newenvironment{corollary}[2][Corollary]{\begin{trivlist}
\item[\hskip \labelsep {\bfseries #1}\hskip \labelsep {\bfseries #2.}]}{\end{trivlist}}
\DeclareMathOperator{\tr}{tr}
\DeclareMathOperator{\rank}{rank}
\DeclareMathOperator{\Span}{span}
\DeclareMathOperator{\row}{row}
\DeclareMathOperator{\col}{col}
\DeclareMathOperator{\range}{range}
\DeclarePairedDelimiterX{\inp}[2]{\langle}{\rangle}{#1, #2}
\DeclareMathOperator{\Proj}{Proj}
\DeclareMathOperator{\trace}{trace}
\newcommand{\Her}{^{\mathrm H}}
\DeclareMathOperator{\diag}{diag}
\makeatletter 
    \newcommand\fcaption{\def\@captype{table}\caption}
\makeatother
\setmarginsrb{3 cm}{2.5 cm}{3 cm}{2.5 cm}{1 cm}{1.5 cm}{1 cm}{1.5 cm}

\title{Lab Report for PHY 1002 Guideline}                             % Title
\author{Name of Author}                               % Author
\date{\today}                                           % Date

\makeatletter
\let\thetitle\@title
\let\theauthor\@author
\let\thedate\@date
\makeatother

\pagestyle{fancy}
\fancyhf{}
\rhead{\theauthor}
\lhead{\thetitle}
\cfoot{\thepage}

\begin{document}

%%%%%%%%%%%%%%%%%%%%%%%%%%%%%%%%%%%%%%%%%%%%%%%%%%%%%%%%%%%%%%%%%%%%%%%%%%%%%%%%%%%%%%%%%

\begin{titlepage}
    \centering
    \vspace*{0.5 cm}
    \includegraphics[scale = 0.75,width=6cm]{CUHK}\\[1.0 cm]   % University Logo
    \textsc{\large The Chinese University of Hong Kong, Shenzhen}\\[2.0 cm]   % University Name
    \textsc{\Large PHY 1002}\\[0.5 cm]               % Course Code
    \textsc{\large Physics Laboratory}\\[0.5 cm]               % Course Name
    \rule{\linewidth}{0.2 mm} \\[0.4 cm]
    { \huge \bfseries \thetitle}\\
    \rule{\linewidth}{0.2 mm} \\[1.5 cm]
    
    \begin{minipage}{0.4\textwidth}
        \begin{flushleft} \large
            \emph{Author:}\\
            \theauthor
            \end{flushleft}
            \end{minipage}~
            \begin{minipage}{0.4\textwidth}
            \begin{flushright} \large
            \emph{Student Number:} \\
            116010000                                   % Your Student Number
        \end{flushright}
    \end{minipage}\\[2 cm]
    
    {\large \thedate}\\[2 cm]
 
    \vfill
    
\end{titlepage}

%%%%%%%%%%%%%%%%%%%%%%%%%%%%%%%%%%%%%%%%%%%%%%%%%%%%%%%%%%%%%%%%%%%%%%%%%%%%%%%%%%%%%%%%%
%%%%%%%%%%%%%%%%%%%%%%%%%%%%%%%%%%%%%%%%%%%%%%%%%%%%%%%%%%%%%%%%%%%%%%%%%%%%%%%%%%%%%%%%%

\tableofcontents
\pagebreak


%%%%%%%%%%%%%%%%%%%%%%%%%%%%%%%%%%%%%%%%%%%%%%%%%%%%%%%%%%%%%%%%%%%%%%%%%%%%%%%%%%%%%%%%%
